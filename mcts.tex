\documentclass[compress, color = usenames, dvipsnames]{beamer}

% theme CentraleSupelec
\usepackage{theme/beamerthemeCS}

% liste de paquets
\usepackage[utf8]{inputenc}
\usepackage[francais]{babel}
\usepackage{xstring}
\usepackage{tikz}
\usetikzlibrary{shapes.arrows,chains,fit, calc,positioning, intersections}
\usepackage{pgfplots}
\usepackage{hyperref}
\usepackage{listings}
\usepackage{verbatim}
\usepackage{hyperref}
\usepackage{multimedia}
\usepackage{ifthen}

\usepackage{xcolor}% http://ctan.org/pkg/xcolor
\usepackage{colortbl}

\usepackage{pifont}
\usepackage{textcomp}
\usepackage{textpos}

\usepackage{soul}
%\usepackage{algorithmic}
\usepackage{algpseudocode}



% première page
\title[Jeu de Go \\ et \\ Exploration d'Arbre par Bandit]{Jeu de Go \\ et \\ Exploration d'Arbre par Bandit}
\subtitle[]{}
\date[]{}
\institute[]{\large CentraleSupélec -- Gif}


\begin{document}

\frame{
  \titlepage
}




\section{Jeu de Go}

\subsection{Pourquoi le Jeu de Go?}

\frame{
  \frametitle{Pourquoi un jeu?}

}

\frame{
  \frametitle{Importance du jeu de Go}

  historique

  nombre de joueurs


}


\subsection{Règle du Jeu}

\frame{
  \frametitle{Règles}

}

\frame{
  \frametitle{Echelle de niveau}

}

\section{Avant l'Exploration d'Arbre par Bandit}

\frame{
  \frametitle{Découpage du plateau}

}
\frame{
  \frametitle{Règles expertes}

}
\frame{
  \frametitle{Alpha beta}

}

\section{Exploration d'Arbre par Bandit}

\subsection{Construction de l'Arbre}


\subsection{Problème de Bandit}

\subsection{Amélioration de l'Algorithme}

\frame{
  \frametitle{RAVE}

}

\frame{
  \frametitle{Deep Learning}

}

\section{Conclusion}



\frame{
  \frametitle{Problématique rencontrée}

  \begin{block}{Problème concret}
    \begin{itemize}
    \item Étant donné une liste de ville et les distances entre ces villes, quel est le plus court chemin qui passe une fois par chaque ville et termine à la ville de départ?
    \end{itemize}
  \end{block}

  \begin{center}
  %\includegraphics[width=0.5\textwidth]{figs/TSP.png}
  \end{center}
  
  C'est le problème du \purple{voyageur de commerce} (en anglais \emph{Traveling Salesman Problem} ou \purple{TSP}).

}



\frame{
  \frametitle{Exemple borne}

\centering

\begin{tikzpicture}

    \node [circle,draw] (a) at (0,0) {a};
    \node [circle,draw] (b) at (3,0) {b};
    \node [circle,draw] (c) at (6,0) {c};
    \node [circle,draw] (d) at (0,-4) {d};
    \node [circle,draw] (e) at (-3,-4) {e};

    \draw [dashed] (a) -- (b) node [pos=0.5]{\textbf{3}};
    \draw [dashed] (a) -- (d) node [pos=0.3]{\textbf{4}};
    \draw [dashed] (a) -- (e) node [pos=0.5]{\textbf{5}};
  
    \draw [dashed] (b) -- (c) node [pos=0.5]{\textbf{3}};
    \draw [dashed] (b) -- (d) node [pos=0.3]{5};
    \draw [dashed] (b) -- (e) node [pos=0.3]{7};
  
    \draw [dashed] (c) -- (d) node [pos=0.5]{\textbf{7}};
    \draw [dashed] (c) -- (e) node [pos=0.3]{10};
  
    \draw [dashed] (d) -- (e) node [pos=0.5]{\textbf{3}};
\end{tikzpicture}

    \begin{block}{Calcul de la borne}
        trajet déjà effectué = $\emptyset$

        $(\underbrace{3+4}_{a}+\underbrace{3+3}_{b}+\underbrace{3+7}_{c}+\underbrace{3+4}_{d}+\underbrace{3+5}_{e})/2=19$
    \end{block}

}

\frame{
  \frametitle{Efficacité de la borne: cas du voyageur de commerce}

  \begin{alertblock}{Remarques}
      \begin{itemize}
          \item Le parcours doit passer par toutes les villes.
          \item Pour chaque ville il faut arriver et repartir.
          \itemSo Au mieux, pour chaque ville, on arrivera et on repartira par les deux arcs les plus petits
        \end{itemize}
  \end{alertblock}
  \onslide<2>{
  \begin{block}{Idée}
  La borne utilise les distances aux deux voisins les plus proches
  \end{block}
  }

}

\frame{
  \frametitle{Algorithme}



  \begin{block}{Algorithme}
      \begin{algorithmic}[1]
	
       \Function{$backtracking$}{$s$}
          \If{ $terminal(s)$}
          \State \Return $verify(s)$
          \EndIf
          \ForAll{ $c \in children(s)$}
          \If{ $backtracking(c)$}
          \State \Return $TRUE$
          \EndIf
          \EndFor
          \State \Return $FALSE$
        \EndFunction
      \end{algorithmic}
  \end{block}
}
\end{document}
