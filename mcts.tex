\documentclass[compress, color = usenames, dvipsnames]{beamer}

% theme CentraleSupelec
\usepackage{theme/beamerthemeCS}

% liste de paquets
\usepackage[utf8]{inputenc}
\usepackage[francais]{babel}
\usepackage{xstring}
\usepackage{tikz}
\usetikzlibrary{shapes.arrows,chains,fit, calc,positioning, intersections}
\usepackage{pgfplots}
\usepackage{hyperref}
\usepackage{listings}
\usepackage{verbatim}
\usepackage{hyperref}
\usepackage{multimedia}
\usepackage{ifthen}

\usepackage{xcolor}% http://ctan.org/pkg/xcolor
\usepackage{colortbl}

\usepackage{pifont}
\usepackage{textcomp}
\usepackage{textpos}

\usepackage{soul}
%\usepackage{algorithmic}
\usepackage{algpseudocode}



% première page
\title[Monte Carlo Tree Search]{Monte Carlo Tree Search}
\subtitle[]{}
\date[ST2 ~-- Gif]{\large ST2 ~-- Gif}
\institute[]{\large CentraleSupélec -- Gif}


\begin{document}

\frame{
  \titlepage
}



\section{Introduction}
\frame{
  \frametitle{Problématique rencontrée}

  \begin{block}{Problème concret}
    \begin{itemize}
    \item Étant donné une liste de ville et les distances entre ces villes, quel est le plus court chemin qui passe une fois par chaque ville et termine à la ville de départ?
    \end{itemize}
  \end{block}

  \begin{center}
  %\includegraphics[width=0.5\textwidth]{figs/TSP.png}
  \end{center}
  
  C'est le problème du \purple{voyageur de commerce} (en anglais \emph{Traveling Salesman Problem} ou \purple{TSP}).

}


\section{Construction de l'Arbre}
\frame{
  \frametitle{Gérer la complexité} 
  
  Beaucoup de problèmes d'optimisation rencontrés sont
  $NP$-complet. Il est donc nécessaire de développer des méthodes pour
  les traiter au mieux.

  \begin{block}{Méthodes de résolution}
    \begin{itemize}
        \item Méthodes exactes \alt<2> { - \red{Cours 9}} {(\textbf{Branch \& Bound}, Programmation Linéaire, \ldots)}        
        \item Heuristiques dédiées \alt<2> { - \blue{Cours 10}} {(\textbf{d'approximation} ou sans garantie de performance)}
        \item Métaheuristiques \alt<2> { - \blue{Cours 10}} {(\textbf{Recuit Simulé}, Tabou, Algorithmes Génétiques, \ldots)}
    \end{itemize}
  \end{block}
}

\section{Simulation Monte Carlo}
\frame{
  \frametitle{Algorithme}



  \begin{block}{Algorithme}
      \begin{algorithmic}[1]
	
       \Function{$backtracking$}{$s$}
          \If{ $terminal(s)$}
          \State \Return $verify(s)$
          \EndIf
          \ForAll{ $c \in children(s)$}
          \If{ $backtracking(c)$}
          \State \Return $TRUE$
          \EndIf
          \EndFor
          \State \Return $FALSE$
        \EndFunction
      \end{algorithmic}
  \end{block}
}

\section{Problème de Bandit}
\frame{
  \frametitle{Efficacité de la borne: cas du voyageur de commerce}

  \begin{alertblock}{Remarques}
      \begin{itemize}
          \item Le parcours doit passer par toutes les villes.
          \item Pour chaque ville il faut arriver et repartir.
          \itemSo Au mieux, pour chaque ville, on arrivera et on repartira par les deux arcs les plus petits
        \end{itemize}
  \end{alertblock}
  \onslide<2>{
  \begin{block}{Idée}
  La borne utilise les distances aux deux voisins les plus proches
  \end{block}
  }

\begin{center}
\begin{tikzpicture}[scale=0.5, every node/.style={transform shape}]]
    \node [circle,draw] (a) at (0,0) {a};
    \node [circle,draw] (b) at (3,0) {b};
    \node [circle,draw] (c) at (6,0) {c};
    \node [circle,draw] (d) at (0,-4) {d};
    \node [circle,draw] (e) at (-3,-4) {e};

    \draw [dashed] (a) -- (b) node [pos=0.5]{\only<1>{3}\only<2>{\textbf{\red{3}}}};
    \draw [dashed] (a) -- (d) node [pos=0.3]{\only<1>{4}\only<2>{\textbf{\red{4}}}};
    \draw [dashed] (a) -- (e) node [pos=0.5]{\only<1>{5}\only<2>{\textbf{\red{5}}}};
  
    \draw [dashed] (b) -- (c) node [pos=0.5]{\only<1>{3}\only<2>{\textbf{\red{3}}}};
    \draw [dashed] (b) -- (d) node [pos=0.3]{5};
    \draw [dashed] (b) -- (e) node [pos=0.3]{7};
  
    \draw [dashed] (c) -- (d) node [pos=0.5]{\only<1>{7}\only<2>{\textbf{\red{7}}}};
    \draw [dashed] (c) -- (e) node [pos=0.3]{10};
  
    \draw [dashed] (d) -- (e) node [pos=0.5]{\only<1>{3}\only<2>{\textbf{\red{3}}}};
\end{tikzpicture}
\end{center}
}

\section{Amélioration de l'Algorithme}
\frame{
  \frametitle{Exemple borne}

\centering

\begin{tikzpicture}

    \node [circle,draw] (a) at (0,0) {a};
    \node [circle,draw] (b) at (3,0) {b};
    \node [circle,draw] (c) at (6,0) {c};
    \node [circle,draw] (d) at (0,-4) {d};
    \node [circle,draw] (e) at (-3,-4) {e};

    \draw [dashed] (a) -- (b) node [pos=0.5]{\textbf{3}};
    \draw [dashed] (a) -- (d) node [pos=0.3]{\textbf{4}};
    \draw [dashed] (a) -- (e) node [pos=0.5]{\textbf{5}};
  
    \draw [dashed] (b) -- (c) node [pos=0.5]{\textbf{3}};
    \draw [dashed] (b) -- (d) node [pos=0.3]{5};
    \draw [dashed] (b) -- (e) node [pos=0.3]{7};
  
    \draw [dashed] (c) -- (d) node [pos=0.5]{\textbf{7}};
    \draw [dashed] (c) -- (e) node [pos=0.3]{10};
  
    \draw [dashed] (d) -- (e) node [pos=0.5]{\textbf{3}};
\end{tikzpicture}

    \begin{block}{Calcul de la borne}
        trajet déjà effectué = $\emptyset$

        $(\underbrace{3+4}_{a}+\underbrace{3+3}_{b}+\underbrace{3+7}_{c}+\underbrace{3+4}_{d}+\underbrace{3+5}_{e})/2=19$
    \end{block}

}
\section{Conclusion}







\end{document}
