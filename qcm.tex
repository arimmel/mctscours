\documentclass{article}

\renewcommand{\labelenumii}{\arabic{enumii}.}
\newcommand{\itembf}{\item[\stepcounter{enumii}\textbf{\arabic{enumii}.}]}

\begin{document} 

\title{Questions Conférence - Monte Carlo Tree Search (MCTS)}
\date{}
\maketitle
\begin{enumerate}
    \item Laquelle(lesquelles) de ces propriétés concernant l'algorithme MCTS est(sont) vraie(s)?
    \begin{enumerate}
        \item L'algorithme explore la totalité de l'arbre des parties possibles.
        \itembf \textbf{L'arbre de recherche est déséquilibré.}
        \itembf \textbf{L'algorithme est itératif.}
        \item L'algorithme est déterministe.
    \end{enumerate}
    \item Quelles sont les étapes de l'algorithme MCTS?
    \begin{enumerate}
        \item Selection, Evaluation, Update.
        \item Descent, Evaluation, Update.
        \item Selection, Expansion, Update, Backpropagation.
        \itembf \textbf{Selection, Expansion, Simulation, Backpropagation.}
    \end{enumerate}
    \item Laquelle(lesquelles) de ces informations concernant le problème du bandit est(sont) vraie(s)?
    \begin{enumerate}
        \item Les distributions de probabilités de chaque bras sont identiques.
        \item Le but est de maximiser les profits.
        \itembf \textbf{Le tirage de chaque bras est indépendant et identiquement distribué.}
        \itembf \textbf{Toute stratégie ne peut obtenir de meilleur performance que O(log(n)) fois la stratégie optimale.}
    \end{enumerate}
\item Laquelle(lesquelles) de ces informations concernant l'algorithme Upper Confidence Bound (UCB) est(sont) vraie(s)?
    \begin{enumerate}
        \itembf \textbf{L'algorithme est basé sur un compromis exploration/exploitation.}
        \item L'algorithme est randomisé.
        \itembf \textbf{L'algorithme est asymptotiquement optimal.}
        \item L'algorithme choisit à chaque étape le bras associé au plus grand écart type.
    \end{enumerate}
\end{enumerate}
\end{document}
